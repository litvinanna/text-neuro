\section{Введение}

С появлением первых секвенированных последовательностей генов и геномов появились попытки описать эти последовательности с помощью математических моделей.

Одни из первых моделей опирались на известные тогда вирусные геномы -- стационарная \cite{garden_markov_1980}, нестационарная Марковская модели \cite{tavare_codon_1989}, скрытые Марковский модели \cite{churchill_stochastic_1989}. 
Давно стало известно, что геномы про- и эукариот неравновесны по вхождениям ди-, три-, тетрануклеотидных последовательностей -- в общем случае k-меров \cite{phillips_mono-_1987}. 

Тем не менее, описать общую структуру генома, не вдаваясь в конкретные последовательности генов и мотивов, до сих пор не удалось.

В настоящее время объем геномных данных огромен. Это не облегчает задачу. Машинное обучение. Классические методы машинного обучения нуждаются в ручном выборе характеристик для обучения. Нейросети придуманы так, что распознавание нужных характеристик встроено в процесс обучения.

Взрыву использования нейронных сетей и глубокого обучения способствовал стремительный рост количества данных, появление новых алгоритмов расчетов и существенное увеличение вычислительных мощностей, особенно с использованием графических процессоров (GPU).
За последние 7 лет использование нейросетей привело к прорывам в областях компьютерного зрения \cite{krizhevsky_imagenet_2012, girshick_region-based_2016, long_fully_2015}, распознавании речи \cite{hannun_deep_2014}, машинного перевода и распознавания естественного языка \cite{wu_googles_2016}.

Первые пионерские исследования в области геномики с применением глубокого обучения были проведены недавно. DeepBind позволяет распознавать особенности последовательностей ДНК, связывающих белки и РНК \cite{alipanahi_predicting_2015}. DeepSEA может выучивать регуляторный код  прямо из последовательности ДНК и данных хроматинового профайлинга \cite{zhou_predicting_2015}. 

С этого момента число применений глубокого обучения в биологических задачах растет очень быстро. Даже если рассматривать подходы, связанные с извлечением информации из последовательности, (еще примеры).

Но никто не использовал их для предсказания нуклеотидов из контекста? Мотивация контекста. Цель.
