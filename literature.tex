\section{Литература}
\subsection{Устройство генома}
С появлением первых секвенированных последовательностей генов и геномов появились попытки описать эти последовательности с помощью математических моделей.

Одни из первых опирались на вирусные геномы -- стационарная \cite{garden_markov_1980}, нестационарная марковская модели \cite{tavare_codon_1989}, скрытые марковский модели \cite{churchill_stochastic_1989}. 

Анализ устройства генома был начат с появлением первых секвенированных последовательностей генов. Анализ частот встречаемости моно-...гексануклеотидов был проведен в 1987 году по 80kbp кодирующих и некодирующих последовательностей генома E.coli \cite{phillips_mono-_1987}. 

В геноме найдено множество ассоциаций различных черт последовательностей \cite{pevzner_nucleotide_1992}.

Сейчас известно, что геномы про- и эукариот неравновесны по вхождениям ди-, три-, тетра-, нуклеотидных последовательностей.




\subsection{Сетки}
Искусственные нейронные сети показали себя как мощный инструмент машинного обучения.


Нейросети широко применяются во многих задачах.
Нас интересуют такие, где идет непосредственная обработка нуклеотидных последовательностей.


Сверточные нейросети применяются для классификации последовательностей, распознавания в них каких либо мотивов.

Классификация последовательностей, в частности предсказание сайтов A $\rightarrow $ I редактирования РНК   \cite{budach_pysster:_2018}.

Сверточная нейронная сетка для предсказания сайтов сплайсинга кольцевых РНК \cite{wang_deep_2019}.

 Предсказание сигнала поли-А \cite{arefeen_deeppasta:_2019} обрабатывают последовательности CNN и lstm.

Распознавагие центромерных последовательностей эукариот \cite{li_identifying_2019} rnn bdrnn .
