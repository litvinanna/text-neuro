%-------------------BASICS-------------------
\documentclass[oneside,final,12pt]{article} %одностороння печать, чистовая версия, размер кегля, класс документа
\usepackage{ucs} 
\usepackage[utf8x]{inputenc} 
\usepackage[T2A, T1]{fontenc} 
\usepackage[english,russian]{babel} %оформление кириллицей (подписи к таблицам и т.д.)
\usepackage{ifpdf}
\ifpdf  %% если используется pdfTEX
\usepackage{cmap}  %поиск по кириллице в готовом pdf
\usepackage[pdftex]{graphicx} %работа с графикой 
\usepackage[unicode=true]{hyperref}
\usepackage{pdfpages}
\else   %% если используется не pdfTEX
\usepackage[dvips]{graphicx}
\fi
\usepackage{verbatim} %comments
\setcounter{tocdepth}{2} % глубина содержания
 
 
%-------------------FORMAT-------------------
\usepackage{vmargin} %размеры полос набора
\setpapersize{A4} %формат бумаги
\setmarginsrb{20mm}{20mm}{20mm}{20mm}{0pt}{0mm}{0pt}{13mm} %размеры полей: левое, верхнее, правое, нижнее, 3*колонтитулы, расстояние между нижним краем нижней строки и нижним краем номера страницы
\usepackage{indentfirst} %красная строка для первого абзаца главы или параграфа
\setlength{\parindent}{0cm} % отступ красной строки
\setlength{\parskip}{2mm}
\sloppy %борьба с залезанием строк на поля путём изменения размеров пробелов
\pagestyle{plain} %включена нумерация страниц 
\renewcommand{\thesection}{\arabic{section}}  %арабская нумерация глав
\usepackage{lineno} %нумерация всех строк для отладки
\usepackage{enumitem} %особенности enumerate
\usepackage{textgreek} % \textalpha греческие буквы не в math mode
\usepackage{setspace} %for setstretch
\usepackage{adjustbox}
\usepackage{rotating}
\usepackage{lscape} 

%-------------------TABLES, FIGURES-------------------
\usepackage{float} %для плавающих картинок и таблиц
%\usepackage{wrapfig} %для плавающих картинок.
\usepackage[font=small]{caption}
\usepackage{booktabs} %отступы в tabular
\usepackage{colortbl} %раскаршивание таблиц
\usepackage{xcolor} %название цветов
\newcommand{\red}[1]{\textcolor{red}{#1}} % выделение красным текста командой \red{}

\definecolor{dark-dark-gray}{gray}{0.3}
\newcommand{\gray}[1]{\textcolor{dark-dark-gray}{#1}}
\definecolor{dark-gray}{gray}{0.4}
\newcommand{\graytable}[0]{\arrayrulecolor{dark-gray}}
\newcommand{\thinrule}[0]{\specialrule{0.3pt}{4pt}{4pt}}
\newcommand{\verythinrule}[0]{\specialrule{0.1pt}{1pt}{1pt}}
\newcommand{\invisiblerule}[0]{\specialrule{0pt}{2pt}{2pt}}

\usepackage{pbox} % для переносов внутри ячейки таблицы
\usepackage{array} %??
\usepackage{longtable}% перенос таблиц на страницах
\usepackage{subcaption} % несколько картинок в одной
\usepackage{multirow} % объединение ячеек
\newcolumntype{R}[1]{>{\raggedleft\arraybackslash}p{#1}}
\usepackage{tabu}

\usepackage{caption}
\captionsetup{%
    ,format=hang
    ,justification=raggedright
    ,singlelinecheck=false
    ,figureposition=bottom
    }

\usepackage{tikz}
\usetikzlibrary{shapes.geometric,shapes.arrows,decorations.pathmorphing}
\usetikzlibrary{matrix,chains,scopes,positioning,arrows,fit}


%-------------------MATH-------------------
\usepackage{amsmath} %дополнительные средства для вёрстки формул
\everymath{\displaystyle}
\usepackage{breqn} %для dmath разбить длинную формулу
%\usepackage{esvect} %vectors

%\usepackage{amscd} %диаграммы
\usepackage{amsfonts} %дополнительные шрифты для формул
\usepackage{amssymb} %дополнительные символы для формул

%-------------------BIBLIOGRAPHY-------------------
\usepackage{cite}
\usepackage[nottoc,notlot,notlof]{tocbibind}


\begin{document}
		%\linenumbers
		\tikzstyle{line}=[thick, ->]
\tikzstyle{dnn}=[draw, circle, minimum width=1cm]

\newcommand{\ohe}[3]{
	\def\step{0.2}
	\def\hlen{#3}
	\foreach \x in {0,...,4}
	\draw[] (#1-\hlen*\step, #2-\step*\x) -- (#1+\hlen*\step,#2-\step*\x);
	
	\foreach \x in {-\hlen,...,\hlen}
		\draw (#1-\step*\x, #2) -- (#1-\step*\x, #2-4*\step); 
	}
	
\newcommand{\oherandom}[4]{
	\def\step{0.2}
	\def\len{#3}
	\def\depth{#4}
	\foreach \x in {0,...,\depth}
	\draw[] (#1-\len*\step*0.5, #2-\step*\x) -- (#1+\len*\step*0.5, #2-\step*\x);
	
	\foreach \x in {0,...,\len}
		\draw (#1+\step*\x-\len*\step*0.5, #2) -- (#1+\step*\x - \len*\step*0.5, #2-\depth*\step); 
	}	
	
	
\newcommand{\flattenout}[3]{
	\def\step{0.1}
	\def\lenflat{#3}
	
	\foreach \x in {0, 1}
	\draw[] (#1-\lenflat*\step*0.5, #2-\step*\x) -- (#1+\lenflat*\step*0.5,#2-\step*\x);
	\foreach \x in {0,...,\lenflat}
		\draw (#1+\step*\x -\lenflat*\step*0.5, #2) -- (#1+\step*\x -\lenflat*\step*0.5, #2-\step);
	}

\newcommand{\out}[2]{
	\def\step{0.5}
	\foreach \x in {0, 1}
		\draw[] (#1-\step*2, #2-\step*\x) -- (#1+\step*2,#2-\step*\x);
	\foreach \x in {-2,...,2}
		\draw (#1+\step*\x, #2) -- (#1+\step*\x, #2-\step); 
	\node[inner sep=0] (out0) at (#1-1.5*\step, #2) {};
	\node[inner sep=0] (out1) at (#1-0.5*\step, #2) {};
	\node[inner sep=0] (out2) at (#1+0.5*\step, #2) {};
	\node[inner sep=0] (out3) at (#1+1.5*\step, #2) {};
}

\tikzstyle{cnn}=[draw, rectangle, minimum height=2em, minimum width = 0.8em]	

%        \begin{titlepage}

\newcommand{\HRule}{\rule{\linewidth}{0.3mm}} % Defines a new command for the horizontal lines, change thickness here

\center

\textbf{\textsc{\Large московский государственный университет} \textsc{\large имени }\textsc{\Large М.В.Ломоносова}}
\\[0.3cm] 
\HRule 
\\[0.3cm]
\textbf{\textsc{\large факультет биоинженерии и биоинформатики}}
\\[4.0cm]

\begin{spacing}{1.4}
{ \LARGE \bfseries  Использование нейронных сетей в задачах предсказания нуклеотидных последовательностей}  \\[1.0cm] 


{\LARGE \bfseries The use of neural networks for nucleotide sequences prediction problems} \\[2.0cm] \end{spacing} 
 
 
\Large \emph{Курсовая работа студентки четвертого курса:}\\
Литвин Анны Валерьевны
\\[3cm]



\begin{flushright} \large
Научные руководители: \\
профессор,\,д.б.н.\,Гельфанд\,М.С.\\
аспирант\,Червонцева\,З.С.\\
\end{flushright}

%\begin{flushleft}\large
%\makebox[1.5in]{\hrulefill} 
%\end{flushleft}



\vfill

{\large Москва \\ 2019}
\end{titlepage}
 \newpage
%        \tableofcontents \newpage
        %\include{abbreviations} \newpage
        \section{Введение}

С появлением первых секвенированных последовательностей генов и геномов появились попытки описать эти последовательности с помощью математических моделей.

Одни из первых моделей опирались на известные тогда вирусные геномы -- стационарная \cite{garden_markov_1980}, нестационарная Марковская модели \cite{tavare_codon_1989}, скрытые Марковский модели \cite{churchill_stochastic_1989}. 
Давно стало известно, что геномы про- и эукариот неравновесны по вхождениям ди-, три-, тетрануклеотидных последовательностей -- в общем случае k-меров \cite{phillips_mono-_1987}. 

Тем не менее, описать общую структуру генома, не вдаваясь в конкретные последовательности генов и мотивов, до сих пор не удалось.

В настоящее время объем геномных данных огромен. Это не облегчает задачу. Машинное обучение. Классические методы машинного обучения нуждаются в ручном выборе характеристик для обучения. Нейросети придуманы так, что распознавание нужных характеристик встроено в процесс обучения.

Взрыву использования нейронных сетей и глубокого обучения способствовал стремительный рост количества данных, появление новых алгоритмов расчетов и существенное увеличение вычислительных мощностей, особенно с использованием графических процессоров (GPU).
За последние 7 лет использование нейросетей привело к прорывам в областях компьютерного зрения \cite{krizhevsky_imagenet_2012, girshick_region-based_2016, long_fully_2015}, распознавании речи \cite{hannun_deep_2014}, машинного перевода и распознавания естественного языка \cite{wu_googles_2016}.

Первые пионерские исследования в области геномики с применением глубокого обучения были проведены недавно. DeepBind позволяет распознавать особенности последовательностей ДНК, связывающих белки и РНК \cite{alipanahi_predicting_2015}. DeepSEA может выучивать регуляторный код  прямо из последовательности ДНК и данных хроматинового профайлинга \cite{zhou_predicting_2015}. 

С этого момента число применений глубокого обучения в биологических задачах растет очень быстро. Даже если рассматривать подходы, связанные с извлечением информации из последовательности, (еще примеры).

Но никто не использовал их для предсказания нуклеотидов из контекста? Мотивация контекста. Цель.

        \section{Литература}
\subsection{Устройство генома}
С появлением первых секвенированных последовательностей генов и геномов появились попытки описать эти последовательности с помощью математических моделей.

Одни из первых опирались на вирусные геномы -- стационарная \cite{garden_markov_1980}, нестационарная марковская модели \cite{tavare_codon_1989}, скрытые марковский модели \cite{churchill_stochastic_1989}. 

Анализ устройства генома был начат с появлением первых секвенированных последовательностей генов. Анализ частот встречаемости моно-...гексануклеотидов был проведен в 1987 году по 80kbp кодирующих и некодирующих последовательностей генома E.coli \cite{phillips_mono-_1987}. 

В геноме найдено множество ассоциаций различных черт последовательностей \cite{pevzner_nucleotide_1992}.

Сейчас известно, что геномы про- и эукариот неравновесны по вхождениям ди-, три-, тетра-, нуклеотидных последовательностей.




\subsection{Сетки}
Искусственные нейронные сети показали себя как мощный инструмент машинного обучения.


Нейросети широко применяются во многих задачах.
Нас интересуют такие, где идет непосредственная обработка нуклеотидных последовательностей.


Сверточные нейросети применяются для классификации последовательностей, распознавания в них каких либо мотивов.

Классификация последовательностей, в частности предсказание сайтов A $\rightarrow $ I редактирования РНК   \cite{budach_pysster:_2018}.

Сверточная нейронная сетка для предсказания сайтов сплайсинга кольцевых РНК \cite{wang_deep_2019}.

 Предсказание сигнала поли-А \cite{arefeen_deeppasta:_2019} обрабатывают последовательности CNN и lstm.

Распознавагие центромерных последовательностей эукариот \cite{li_identifying_2019} rnn bdrnn .

        \section{Методы}
\subsection{Методы кодировки последовательности}
Все нуклеотидные последовательности были закодированны в one-hot-encoded векторы (единичные нуклеотиды) и матрицы (последовательности), где каждому нуклеотиду соответствует один из четырехмерных векторов (0, 0, 0, 1), (0, 0, 1, 0), (0, 1, 0, 0), (1, 0, 0, 0). Это позволяет добиться независимого влияния нуклеотидов на предсказание и используется в категориальных предсказаниях.

\subsection{Использованные функции}

В качестве функции активации использовавалась функция softmax и relu (уравнение \ref{eq:softmax}).
\begin{align} \label{eq:softmax}
softmax(x) &= exp(x - \max_{axis}(x)) \\
relu(x) &= max(x, 0)
\end{align}

Функция потерь во всех архитектурах -- категориалная кроссэнтропия (уравнение \ref{eq:cross}).
\begin{equation} \label{eq:cross}
categorical\_crossentropy(y_{pred}, y_{true}) = -\sum_{x}{y_{pred}(x)\log(y_{true}(x))}
\end{equation}

Для статистического сравнения выборок использовался критерий Манна-Уитни.

\subsection{Простейшие нейросетевые модели}
\subsubsection{Построение выборки контекстов}

Последовательности выбирались из генома  Escherichia coli (Escherichia coli str. K-12 substr. MG1655, сборка GCF 000005845.2).

Выборка представляет собой набор контекстов (предикторных областей) определенной длины и соответственных нуклеотидов для предсказания.

Выборка состояла из нескольких частей --  для обучения алгоритма (train), валидации в процессе обучения (validate), тестирования (test). При этом эти части были взяты из непересекающихся геномных областей, чтобы предотвратить выучивание алгоритмом последовательности нуклеотидов.

Для статистической проверки каждого метода было построено 30 выборок. Во всех 30 выборках области, соответствующие тренировочной и валидационной части, были разные.

Размер тренировочной выборки обычно составлял 100,000 нуклеотидов, валидационной и тренирочной -- по одной десятой, соответственно 10,000 и 10,000.

Предикторные области (контексты) для каждого нуклеотида находились с 5' конца нуклеотида, их размер варьировал -- 3, 6, 12, 24 нуклеотида. Также создавались выборки со предикторной областью, сдвинутой относительно предсказываемого нуклеотида в 5' сторону на 1, 2, 3, 6, 12, 50 нуклеотидов.

Все предикторные обасти и предсказываемые нуклеотиды были закодированы в виде one-hot-encoded векторов и матриц.

\subsubsection{Архитектура нейронных сетей}
Все нейронные сети были реализованы с помощью библиотек Keras\cite{chollet_keras_2015} , TensorFlow.
Для работы использовалось несколько архитектур и типов слоев.

\begin{figure}[h] % picture
	\centering
	

\begin{tikzpicture}
	\ohe{0}{0.3}{3}
	\flattenout{0}{-0.9}{24}
	\out{0}{-1.6}	
	
	\def\x{4}
	\node (a) at (\x, -0) {двумерные данные};
	\node (a) at (\x, -1) {одномерные данные};
	\node (a) at (\x, -2) {выходные вероятности};
	
	\def\x{8}
	\draw (\x, 0) circle (0.4);
	\node[cnn] (x) at (\x, -1) {c};
	
	\def\x{12}
	\node (a) at (\x, 0) {полносвязный нейрон};
	\node (a) at (\x, -1) {сверточный нейрон};

\end{tikzpicture}
	\caption{Условные обозначения на схемах архитертур нейронныых сетей.}
	\label{fig:legend}	
\end{figure}

\begin{figure*}[h] % two pictures
	\centering
	\begin{subfigure}[t]{0.55\linewidth}
		\begin{tikzpicture}
	\ohe{0}{1.1}{6}
	\draw[line] (0, 0.2) -- (0, 0.05);
	\flattenout{0}{0}{48}
	\out{0}{-3}
	\foreach \x in {0, ..., 3}{
		\draw[line] (\x-1.5, -0.2) -- (\x-1.5,-1);
		\draw (\x-1.5, -1.5) circle (0.4);
		\draw[line] (\x-1.5, -2) -- (out\x);
		}
\end{tikzpicture}
		\caption{{\bfseries DNN model 1} \\*
		Простейшая полносвязная модель состоит из слоя, выпрямляющего начальные данные, четырех полносвязных нейронов с функцией активации softmax.}
		\label{fig:dnn_1_scheme}
	\end{subfigure}
	\begin{subfigure}[t]{0.40\linewidth}
		\begin{tikzpicture}
	\ohe{0}{1.5}{6}
	\flattenout{0}{0.3}{48}
	\out{0}{-4}
	
	\foreach \x in {0, ..., 3}{
		\draw[line] (\x-1.5, 0) -- (\x-1.5,-0.5);
		}
	\foreach \x in {0, ..., 3}{
		\draw (\x-1.5, -1) circle (0.4);
		\draw (\x-1.5, -2.5) circle (0.4);
		
		\draw[line] (\x-1.5, -3) -- (out\x);
		
		\foreach \xz in {0,...,3}{
			\draw[] (\x-1.5, -1.4) -- (\xz-1.5, -2.1);}
	}	
\end{tikzpicture}		
		\caption{{\bfseries DNN model 2} \\*
		Более сложная полносвязная модель включается в себя два слоя нейронов.
		}
		\label{fig:dnn_2_scheme}
	\end{subfigure}
	\caption{{\bfseries Архитектура полносвязныx моделей.} \\*}
	
	\label{fig:dnn_scheme}
	
\end{figure*}

\begin{figure*}[h] % two pictures
	\centering
	\begin{subfigure}[t]{0.3\linewidth}
		\begin{tikzpicture}
	\ohe{-0.1}{0.8}{6}
		\draw[ultra thick] (-1.3,0) rectangle (-0.7,0.8);
		\draw[ultra thick, ->] (-0.7, 0.4) -- (-0.5,0.4);
		
	\draw[line] (0-1, 0) -- (0-1,-1);
	\foreach \x in {1, 2}{
		\draw[line] (\x-1, -0.2) -- (\x-1,-1);
		
		}
	\foreach \x in {0, ..., 2}
		\node[cnn] (x) at (\x-1, -1.5) {c};
		
	\foreach \x in {0,..., 2}{
		\draw[line] (\x-1, -2.0) -- (\x-1,-2.3); }
		
	\oherandom{0}{-2.5}{10}{3}
	
	\flattenout{0}{-3.2}{30}
	
	\foreach \x in {0,..., 3}{
		\draw[line] (\x-1.5, -3.5) -- (\x-1.5,-4); }


	\out{0}{-6}	
	\foreach \x in {0, ..., 3}{
		\draw (\x-1.5, -4.5) circle (0.4);
		\draw[line] (\x-1.5, -5) -- (out\x);	
	}	

\end{tikzpicture}
		\caption{{\bfseries CNN model 1 (kernel=3)*3} \\*
		Минимальный размер фильтра $3\times4$ был выбран исходя из кодонной структуры б\'{о}льшей части генома -- кодирующей.}
		\label{fig:cnn_1_scheme}
	\end{subfigure}
	\begin{subfigure}[t]{0.3\linewidth}
		\input{tables/cnn_2}		
		\caption{{\bfseries CNN model 1 (kernel=6)*3} \\*
		Размер фильтра в сверточном слое может быть увеличен до $6\times4$, вследствие чего уменьшается размерность выхода и число параметров модели.
		}
		\label{fig:cnn_2_scheme}
	\end{subfigure}
	\begin{subfigure}[t]{0.3\linewidth}
		\input{tables/cnn_3}		
		\caption{{\bfseries CNN model 1 (kernel=3)*3 stride=3} \\* 
		Шаг сверточного фильтра может быть увеличен до 3, что может имитировать распознавание кодонной тринуклеотидной структуры кодирующих частей генома.
		}
		\label{fig:cnn_3_scheme}
	\end{subfigure}
	\caption{{\bfseries Архитектура некоторых сверточных моделей.} \\*  Простейшая сверточная модель состоит из трех сверточных нейронов, выход которых представляет из себя матрицу высотой 3, выпрямляющего слоя, четырех полносвязных нейронов с функцией активации softmax.}

	
	\label{fig:cnn_schemes}	
\end{figure*}

\begin{figure}[h] % picture
	\centering
	\begin{tikzpicture}
	\oherandom{0}{1}{12}{4}
	\out{0}{-5.5}
	\draw[line](0, 0) -- (0, -1);
	\draw (-2,-2) rectangle (2, -1);
	\node (x) at (0, -1.5) {LSTM};
	\draw[line] (-2, -1.5) -- (-2.5, -1.5) -- (-2.5, -0.5) -- (2.5, -0.5) -- (2.5, -1.5) -- (2, -1.5) ;
	\draw[line](0, -2) -- (0, -2.5);
	
	\oherandom{0}{-2.5}{12}{1}
%	\node (x) at (2.5, -2.5) {hidden states output};
	
	\foreach \x in {0, ..., 3}{
		\draw[line] (\x-1.5, -3) -- (\x-1.5,-3.5);
		\draw (\x-1.5, -4) circle (0.4);
		\draw[line] (\x-1.5, -4.5) -- (out\x);
		}
\end{tikzpicture}
	\caption{{\bfseries Архитектура рекуррентной модели.} \\* }
	\label{fig:rnn_scheme}	
\end{figure}

{\bfseries Полносвязные модели} (DNN model 1 - рисунок \ref{fig:dnn_1_scheme}). Первая простейшая полносвязная модель состоит из входного слоя, принимающего нуклеотидный контекст, слоя, уплощающего данные в один вектор, четырех полносвязных нейронов с функцией активации softmax, выходом которых являются вероятности для четырех выходных букв. Число параметров модели $16n + 4$, где $n$ -- размер контекста.

(DNN model 2 -- рисунок \ref{fig:dnn_2_scheme}). В полносвязную модель добавлен второй слой нейронов. Число параметров модели $16n + 24$, где $n$ -- размер контекста.


{\bfseries Сверточные модели} (CNN model 1 - рисунок \ref{fig:cnn_schemes}).
Простейшая сверточная модель состоит из слоя сверточных нейронов (с функцией активации relu), который работает непосредственно с матрицей контекста, далее выход свертки уплощается в вектор и подается в слой из 4 решающих выходных полносвязных нейронов (с функцией активации softmax). Конфигурация сверточного слоя может быть различной. Мы исследовали комбинации из разного числа нейронов, с разным размером фильтра (kernel), которые обрабатывают контекст с различным шагом (stride, по умолчанию шаг сверточного фильтра равен 0).

Были использованы следующие конфигурации сверточного слоя: \begin{enumerate}
		\item (kernel = 3)*3, три сверточных фильтра размером 3
		\item (kernel = 6)*3, три сверточных фильтра размером 6
		\item (kernel = 3)*3 stride = 3, три сверточных фильтра размером 3, которые обрабатывают матрицу с шагом 3.
	\end{enumerate}
 
{\bfseries Рекуррентыне модели } (RNN model 1 - рисунок \ref{fig:rnn_scheme}) Рекуррентная модель состояла из одного LSTM (long short-term memory) слоя, который содержал разное число скрытых состояний, и выходного слоя полносвязных нейронов с функцией активации softmax. 


\subsubsection{Процесс обучения}
Все архитектуры компилировались с использованием оптимизатора Adam \cite{kingma_adam:_2014} с параметрами по умолчанию (learning rate = 0.01). Во время обучения контролировалась точность предсказания на валидационной выборке, с прекращением роста точности обучение останавливалось.


\subsection{Deep Image Prior}
Deep Image Prior --  нейронная сеть с архитектурой автоэнкодера с пробросочными соединениями, подробно описана в \cite{ulyanov_deep_2018}

Данная архитектура была адаптирована для геномной последовательности, закодированной в виде one-hot-encoded матрицы соответственного размера $n\times 4$, где $n$ – длина обрабатываемой геномной области.
В данной архитектуре двумерные функции свертки, пулинга, нормализации, апсемплинга были заменены на соответствующие одномерные аналоги.
Суть подхода закллючается в следующем:

\begin{enumerate}
	\item Выбиралась геномная область длиной 500,000 нуклеотидов. Такой размер области позволял наиболее эффективно проводить расчеты.
	\item На области случайно равномерно выбиралось  10\% нуклеотидов, которые далее предсказывались, которые обозначаются как тест или маска.
	\item Нейронная сеть обучалась получать из случайно сгенерированного массива чисел целевую геномную последовательность. Функция потерь при этом не учитывала тестовые (маскированные) нуклеотиды.
	\item Когда функция потерь достигала низких значений, обучение останавливалось. Проверялось, что же предсказывает модель на месте замаскированных нуклеотидов.
\end{enumerate} 




        \section{Результаты}
\newcommand{\mannwhitni}{{\footnotesize Значимость отличия выборок по критерию Манна-Уитни: ns $P>0.05$, * $P\leq 0.05$, ** $P\leq 0.01$, *** $P\leq 0.001$}}

\paragraph{Базовые частотные модели} Наши предикторные данные мы обработали разными статистическими методами для получения точностей для сравнения. Были использованы простейшие модели -- частотные, Марковские модели от 1 до 11 порядка. Число параметров марковской модели 11 порядка приближается к числу нуклеотидов в геноме \emph{E.coli}.

Точность предсказания растет в повышением порядка модели, но и число параметров растет экспоненциально. Также модель более высокого порядка специфична к геному организма, на котором она обучена, так как предикторные области большой длины для марковской модели могут не встречаться в геноме, и происходит уже выучивание паттернов последовательности генома.

\begin{table}[h]
\label{table:baselines} \graytable
\caption{Точности предсказания, полученные различными математическими моделями.}
\newcolumntype{M}{>{$}l<{$}}
\begin{tabular}{l >{$}c<{$} >{$}l<{$}}
	Метод &  \text{Точность, \%} & \text{Число параметров} \\
	\thinrule
	Предсказание по частотам во всем геноме & 25.7 & 4\\
	 \verythinrule
	по частоте в предикторной области 12 нуклеотидов & 26.3 & 4\\
	\verythinrule
	по частоте в предикторной области 24 нуклеотидов & 26.5 & 4\\
	\verythinrule
	марковская модель 2 порядка & 30.3 &  4^2 = 16 \\
	\verythinrule
	марковская модель 3 порядка & 32.4 &  4^3 = 64\\
	\verythinrule
	марковская модель 4 порядка & 33.4 &  4^4 = 256\\
	\verythinrule
	марковская модель 5 порядка & 34.0 &  4^5 = 1,024\\
	\verythinrule
	марковская модель 6 порядка & 34.4 &  4^6 = 4,096\\
	\verythinrule
	марковская модель 7 порядка & 35.0 &  4^7 = 16,384\\
	\verythinrule
	марковская модель 8 порядка & 36.0 &  4^8 = 65,536\\
	\verythinrule
	марковская модель 9 порядка & 38.9 &  4^9 = 262,144\\
	\verythinrule
	марковская модель 10 порядка & 46.2 &  4^{10} = 1,048,576 \\
	\verythinrule
	марковская модель 11 порядка & 57.0 &  4^{11} \approx \text{геном }  E.coli  \\
	
\end{tabular}
	
\end{table}



\paragraph{Зависимость от размера предикторной области.} Была исследована зависимость качества предсказания нуклеотида от размера использованной предикторной области. Простейшие архитектуры нейронных сетей (две полносвязных модели, сверточная с разным размером сверточного фильтра) были обучены и протестированы на 30 датасетах. Полученные распределения точности приведены на рисунке \ref{fig:size}. Для всех моделей наблюдается увеличение точности при увеличении размера предикторной области, что подтверждается статистическим критерием Манна-Уитни.

Предикторная область большего размера содержит больше информации, от которой может зависеть следующий нуклеотид. Тем не менее при геометрическом увеличении размера области качество растет практически линейно. Из этого можно сделать вывод о том, что более далекие нуклеотиды слабее влияют на предсказываемую позицию.


\begin{figure*}[h] % two pictures
	\centering
	\begin{subfigure}[t]{0.48\linewidth}
%		\caption{{\bfseries DNN model 1} \\* Полносвязная однослойная модель}
		\includegraphics[width = \textwidth]{pics/dnn_model_1_all_runs_p1_ecoli_100000_10000_all_0.png}
		\label{fig:alpha}
	\end{subfigure}
	\begin{subfigure}[t]{0.48\linewidth}
%		\caption{mm}
		\includegraphics[width = \textwidth]{pics/dnn_model_2_all_runs_p1_ecoli_100000_10000_all_0.png}
			\label{fig:alpha}
	\end{subfigure}
	\begin{subfigure}[]{0.48\linewidth}
%		\caption{{\bfseries CNN model 1} \\* Сверточная однослойная модель }
		\includegraphics[width = \textwidth]{pics/cnn_model_1_all_runs_p1_ecoli_100000_10000_all_0.png}
		\label{fig:beta}
	\end{subfigure}
	\begin{subfigure}[]{0.48\linewidth}
%		\caption{{\bfseries CNN model 1} \\* Сверточная однослойная модель }
		\includegraphics[width = \textwidth]{pics/cnn_model_2_all_runs_p1_ecoli_100000_10000_all_0.png}
		\label{fig:cnn_2_predictor}
	\end{subfigure}
	\caption{{\bfseries Зависимость точности предсказания от размера предикторной области для различных архитектур.} \\*
	По горизонтальной оси обозначен размер области. По вертикальной оси показано распределение точностей обученной модели в сете из 30 запусков с различными наборами данных.}
	
	\label{fig:size}
	
\end{figure*}

\paragraph{Зависимость от отступа.} Была исследована зависимость качества предсказания от расстояния между предикторной областью и предсказываемым нуклеотидом для двух моделей -- полносвязной и сверточной. 

С увеличением отступа качество предсказания падает, причем резко. Это подтверждает то, что в простых моделях (полносвязных и сверточных с небольшим числом параметров) предсказание основывается на ближайших в предсказываемому нуклеотидах.

\begin{figure*}[h] % two pictures
	\centering
	\begin{subfigure}[t]{0.47\linewidth}
		\includegraphics[width = \textwidth]{pics/dnn_model_1_all_runs_p1_ecoli_100000_10000_12_all.png}
		\caption{{\bfseries DNN model 1} \\*
		Полносвязная однослойная модель. Размер предикторной области 12.
		}
		\label{fig:dnn_shift}
	\end{subfigure}
	\begin{subfigure}[t]{0.47\linewidth}
		\includegraphics[width = \textwidth]{pics/cnn_model_1_all_runs_p1_ecoli_100000_10000_6_all.png}
		\caption{{\bfseries CNN model 1 (kernel = 3)} \\*
		Сверточная однослойная модель. Размер предикторной области 6.
		}
		\label{fig:cnn_shift}
	\end{subfigure}
	\caption{{\bfseries Зависимость точности предсказания от расстояния между предикторной областью и предсказываемым нуклеотидом (от отступа)  для различных архитектур.} \\*
	По горизонтальной оси обозначен размер отступа. По вертикальной оси показано распределение точностей обученной модели в 30 запусках с различными наборами данных. Горизонтальная линия отмечает точность случайного предсказания 25\%.}	
	\label{fig:shift}
	
\end{figure*}
 
 \paragraph{Сравнение полносвязных моделей.} Мы сравнили между собой полносвязные модели -- с одним (DNN model 1) и двумя слоями (DNN model 2)  на двух типах данных -- предсказание по предикторной области 12 и 24 (рисунок \ref{fig:dnn_test}). Статистически значимой разницы между моделями не наблюдалось.
 
 Полносвязные модели по построению не могут должным образом использовать информацию о близком расположении и последовательности нуклеотидов. В таких моделях предсказание, большей степени, основывается на нуклеотидном составе предикторной области.
 
\begin{figure}[h] % picture
	\centering
	\includegraphics[width = 0.6\textwidth]{pics/dnn_models_all_runs_p1_ecoli_100000_10000_12_0.png}
	\caption{{\bfseries Сравнение полносвязных моделей.} \\* 
		На рисунке показано распределение качестве предсказания полносвязных моделей на двух вариантах данных -- с предикторной областью 12 и 24. В обоих случаях разница в качестве предсказания статистически не значима. \\
		   \mannwhitni }
	\label{fig:dnn_test}	
\end{figure}


\paragraph{Сравнение сверточных моделей.} Мы исследовали несколько вариантов конфигурации сверточного слоя. Результаты тестов приведены на рисунке \ref{fig:cnn_test}.

\input{tables/cnn_comparison}


\paragraph{Рекурретные модели.}
\begin{figure}[h] % picture
	\centering
	\includegraphics[width = 0.6\textwidth]{pics/rnn_models_all_runs_p1_ecoli_100000_10000_50_0.png}
	\caption{{\bfseries Сравнение рекуррентных моделей.} \\* 
		   \mannwhitni }
	\label{fig:rnn_test}	
\end{figure}

	 \newpage
%        %
%\begin{figure}[h] % picture
%	\centering
%	\includegraphics[width = 0.1\textwidth]{}
%	\caption{caption}
%	\label{fig:plasmid}	
%\end{figure}
%
%
%
%\begin{figure*}[h] % two pictures
%	\centering
%	\begin{subfigure}[t]{0.1\linewidth}
%		\includegraphics[width = \textwidth]{}
%		\caption{caption 1}\label{fig:alpha}
%	\end{subfigure}
%	\begin{subfigure}[t]{0.1\linewidth}
%		\includegraphics[width = \textwidth]{}
%		\caption{caption 2}\label{fig:beta}
%
%	\end{subfigure}
%	\caption{great caption}
%	\label{fig:structs}
%	
%\end{figure*}
%
%
%
%\begin{table}[p]
%	\small
%	\caption{caption}
%	\label{table:strains}
%	\begin{tabular}{ p p p }
%
%		Название штамма & Описание & Источник \\ 
%
%		W303 & MAT\textbf{a} ade2-101 his3-11 trp1-1 ura3-52 can1-100 leu2-3,112, GAL, psi+ & $^1$ \\  
%
%
%		
%		блабла & блабал & jdsjf \\
%
%	\end{tabular}
%	
%\end{table}
%
%


        \footnotesize \bibliographystyle{mybib1.bst}  % стилевой файл для оформления по ГОСТу  utf8gost705u, ieeetr для курсовой, cell  -- черновик
        \bibliography{neuro.bib}
\end{document}
