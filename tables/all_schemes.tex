\begin{figure}[h] % picture
	\centering
	

\begin{tikzpicture}
	\ohe{0}{0.3}{3}
	\flattenout{0}{-0.9}{24}
	\out{0}{-1.6}	
	
	\def\x{4}
	\node (a) at (\x, -0) {двумерные данные};
	\node (a) at (\x, -1) {одномерные данные};
	\node (a) at (\x, -2) {выходные вероятности};
	
	\def\x{8}
	\draw (\x, 0) circle (0.4);
	\node[cnn] (x) at (\x, -1) {c};
	
	\def\x{12}
	\node (a) at (\x, 0) {полносвязный нейрон};
	\node (a) at (\x, -1) {сверточный нейрон};

\end{tikzpicture}
	\caption{Условные обозначения на схемах архитертур нейронныых сетей.}
	\label{fig:legend}	
\end{figure}

\begin{figure*}[h] % two pictures
	\centering
	\begin{subfigure}[t]{0.55\linewidth}
		\begin{tikzpicture}
	\ohe{0}{1.1}{6}
	\draw[line] (0, 0.2) -- (0, 0.05);
	\flattenout{0}{0}{48}
	\out{0}{-3}
	\foreach \x in {0, ..., 3}{
		\draw[line] (\x-1.5, -0.2) -- (\x-1.5,-1);
		\draw (\x-1.5, -1.5) circle (0.4);
		\draw[line] (\x-1.5, -2) -- (out\x);
		}
\end{tikzpicture}
		\caption{{\bfseries DNN model 1} \\*
		Простейшая полносвязная модель состоит из слоя, выпрямляющего начальные данные, четырех полносвязных нейронов с функцией активации softmax.}
		\label{fig:dnn_1_scheme}
	\end{subfigure}
	\begin{subfigure}[t]{0.40\linewidth}
		\begin{tikzpicture}
	\ohe{0}{1.5}{6}
	\flattenout{0}{0.3}{48}
	\out{0}{-4}
	
	\foreach \x in {0, ..., 3}{
		\draw[line] (\x-1.5, 0) -- (\x-1.5,-0.5);
		}
	\foreach \x in {0, ..., 3}{
		\draw (\x-1.5, -1) circle (0.4);
		\draw (\x-1.5, -2.5) circle (0.4);
		
		\draw[line] (\x-1.5, -3) -- (out\x);
		
		\foreach \xz in {0,...,3}{
			\draw[] (\x-1.5, -1.4) -- (\xz-1.5, -2.1);}
	}	
\end{tikzpicture}		
		\caption{{\bfseries DNN model 2} \\*
		Более сложная полносвязная модель включается в себя два слоя нейронов.
		}
		\label{fig:dnn_2_scheme}
	\end{subfigure}
	\caption{{\bfseries Архитектура полносвязныx моделей.} \\*}
	
	\label{fig:dnn_scheme}
	
\end{figure*}

\begin{figure*}[h] % two pictures
	\centering
	\begin{subfigure}[t]{0.3\linewidth}
		\begin{tikzpicture}
	\ohe{-0.1}{0.8}{6}
		\draw[ultra thick] (-1.3,0) rectangle (-0.7,0.8);
		\draw[ultra thick, ->] (-0.7, 0.4) -- (-0.5,0.4);
		
	\draw[line] (0-1, 0) -- (0-1,-1);
	\foreach \x in {1, 2}{
		\draw[line] (\x-1, -0.2) -- (\x-1,-1);
		
		}
	\foreach \x in {0, ..., 2}
		\node[cnn] (x) at (\x-1, -1.5) {c};
		
	\foreach \x in {0,..., 2}{
		\draw[line] (\x-1, -2.0) -- (\x-1,-2.3); }
		
	\oherandom{0}{-2.5}{10}{3}
	
	\flattenout{0}{-3.2}{30}
	
	\foreach \x in {0,..., 3}{
		\draw[line] (\x-1.5, -3.5) -- (\x-1.5,-4); }


	\out{0}{-6}	
	\foreach \x in {0, ..., 3}{
		\draw (\x-1.5, -4.5) circle (0.4);
		\draw[line] (\x-1.5, -5) -- (out\x);	
	}	

\end{tikzpicture}
		\caption{{\bfseries CNN model 1 (kernel=3)*3} \\*
		Минимальный размер фильтра $3\times4$ был выбран исходя из кодонной структуры б\'{о}льшей части генома -- кодирующей.}
		\label{fig:cnn_1_scheme}
	\end{subfigure}
	\begin{subfigure}[t]{0.3\linewidth}
		\input{tables/cnn_2}		
		\caption{{\bfseries CNN model 1 (kernel=6)*3} \\*
		Размер фильтра в сверточном слое может быть увеличен до $6\times4$, вследствие чего уменьшается размерность выхода и число параметров модели.
		}
		\label{fig:cnn_2_scheme}
	\end{subfigure}
	\begin{subfigure}[t]{0.3\linewidth}
		\input{tables/cnn_3}		
		\caption{{\bfseries CNN model 1 (kernel=3)*3 stride=3} \\* 
		Шаг сверточного фильтра может быть увеличен до 3, что может имитировать распознавание кодонной тринуклеотидной структуры кодирующих частей генома.
		}
		\label{fig:cnn_3_scheme}
	\end{subfigure}
	\caption{{\bfseries Архитектура некоторых сверточных моделей.} \\*  Простейшая сверточная модель состоит из трех сверточных нейронов, выход которых представляет из себя матрицу высотой 3, выпрямляющего слоя, четырех полносвязных нейронов с функцией активации softmax.}

	
	\label{fig:cnn_schemes}	
\end{figure*}

\begin{figure}[h] % picture
	\centering
	\begin{tikzpicture}
	\oherandom{0}{1}{12}{4}
	\out{0}{-5.5}
	\draw[line](0, 0) -- (0, -1);
	\draw (-2,-2) rectangle (2, -1);
	\node (x) at (0, -1.5) {LSTM};
	\draw[line] (-2, -1.5) -- (-2.5, -1.5) -- (-2.5, -0.5) -- (2.5, -0.5) -- (2.5, -1.5) -- (2, -1.5) ;
	\draw[line](0, -2) -- (0, -2.5);
	
	\oherandom{0}{-2.5}{12}{1}
%	\node (x) at (2.5, -2.5) {hidden states output};
	
	\foreach \x in {0, ..., 3}{
		\draw[line] (\x-1.5, -3) -- (\x-1.5,-3.5);
		\draw (\x-1.5, -4) circle (0.4);
		\draw[line] (\x-1.5, -4.5) -- (out\x);
		}
\end{tikzpicture}
	\caption{{\bfseries Архитектура рекуррентной модели.} \\* }
	\label{fig:rnn_scheme}	
\end{figure}