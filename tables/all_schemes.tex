\begin{figure}[h] % picture
	\centering
	

\begin{tikzpicture}
	\ohe{0}{0.3}{3}
	\flattenout{0}{-0.9}{24}
	\out{0}{-1.6}	
	
	\def\x{4}
	\node (a) at (\x, -0) {двумерные данные};
	\node (a) at (\x, -1) {одномерные данные};
	\node (a) at (\x, -2) {выходные вероятности};
	
	\def\x{8}
	\draw (\x, 0) circle (0.4);
	\node[cnn] (x) at (\x, -1) {c};
	
	\def\x{12}
	\node (a) at (\x, 0) {полносвязный нейрон};
	\node (a) at (\x, -1) {сверточный нейрон};

\end{tikzpicture}
	\caption{Условные обозначения на схемах архитертур нейронныых сетей.}
	\label{fig:legend}	
\end{figure}

\begin{figure*}[h] % two pictures
	\centering
	\begin{subfigure}[t]{0.55\linewidth}
		\begin{tikzpicture}
	\ohe{0}{1.1}{6}
	\draw[line] (0, 0.2) -- (0, 0.05);
	\flattenout{0}{0}{48}
	\out{0}{-3}
	\foreach \x in {0, ..., 3}{
		\draw[line] (\x-1.5, -0.2) -- (\x-1.5,-1);
		\draw (\x-1.5, -1.5) circle (0.4);
		\draw[line] (\x-1.5, -2) -- (out\x);
		}
\end{tikzpicture}
		\caption{{\bfseries} \\*}		
		\label{fig:dnn_1_scheme}
	\end{subfigure}
	\begin{subfigure}[t]{0.40\linewidth}
		\begin{tikzpicture}
	\ohe{0}{1.1}{6}
	\draw[line] (0, 0.2) -- (0, 0.05);
	\flattenout{0}{0}{48}
	\out{0}{-3}
	
	\foreach \x in {0, ..., 3}{
		\draw[line] (\x-1.5, -0.2) -- (\x-1.5,-0.5);
		}
	\foreach \x in {0, ..., 3}{
		\draw (\x-1.5, -1) circle (0.4);		
		\foreach \xz in {0,...,3}{
			\draw[] (\x-1.5, -1.4) -- (\xz-1.5, -1.8);}
			
		\draw (\x-1.5, -2.2) circle (0.4);
		\draw[line] (\x-1.5, -2.7) -- (out\x);
	}	
\end{tikzpicture}		
		\caption{{\bfseries} \\*}
		\label{fig:dnn_2_scheme}
	\end{subfigure}
	\caption{{\bfseries Архитектура полносвязныx моделей.} \\*
	(a) Простейшая полносвязная модель состоит из слоя, выпрямляющего начальные данные, четырех полносвязных нейронов с функцией активации softmax. (b) Более сложная полносвязная модель включается в себя два слоя нейронов, первый с активацией ReLU, второй с softmax.}
	
	\label{fig:dnn_scheme}
	
\end{figure*}

\begin{figure*}[h] % two pictures
	\centering
	\begin{subfigure}[t]{0.3\linewidth}
		\begin{tikzpicture}
	\ohe{-0.1}{0.8}{6}
		\draw[ultra thick] (-1.3,0) rectangle (-0.7,0.8);
		\draw[ultra thick, ->] (-0.7, 0.4) -- (-0.5,0.4);
		
	\draw[line] (0-1, 0) -- (0-1,-1);
	\foreach \x in {1, 2}{
		\draw[line] (\x-1, -0.2) -- (\x-1,-1);
		
		}
	\foreach \x in {0, ..., 2}
		\node[cnn] (x) at (\x-1, -1.5) {c};
		
	\foreach \x in {0,..., 2}{
		\draw[line] (\x-1, -2.0) -- (\x-1,-2.3); }
		
	\oherandom{0}{-2.5}{10}{3}
	
	\flattenout{0}{-3.2}{30}
	
	\foreach \x in {0,..., 3}{
		\draw[line] (\x-1.5, -3.5) -- (\x-1.5,-4); }


	\out{0}{-6}	
	\foreach \x in {0, ..., 3}{
		\draw (\x-1.5, -4.5) circle (0.4);
		\draw[line] (\x-1.5, -5) -- (out\x);	
	}	

\end{tikzpicture}
		\caption{{\bfseries CNN model 1 (kernel=3)*3} \\*}
		\label{fig:cnn_1_scheme}
	\end{subfigure}
	\begin{subfigure}[t]{0.3\linewidth}
		\begin{tikzpicture}
	\ohe{-0.1}{0.8}{6}
		\draw[ultra thick] (-1.3,0) rectangle (-0.1,0.8);
		\draw[ultra thick, ->] (-0.1, 0.4) -- (0.3,0.4);
		
	\draw[line] (0-1, 0) -- (0-1,-1);
	\foreach \x in {1, 2}{
		\draw[line] (\x-1, -0.2) -- (\x-1,-1);
		
		}
	\foreach \x in {0, ..., 2}
		\node[cnn] (x) at (\x-1, -1.5) {c};
		
	\foreach \x in {0,..., 2}{
		\draw[line] (\x-1, -2.0) -- (\x-1,-2.3); }
		
	\oherandom{0}{-2.5}{7}{3}
	
	\flattenout{0}{-3.2}{21}
	
	\foreach \x in {0,..., 3}{
		\draw[line] (\x-1.5, -3.5) -- (\x-1.5,-4); }


	\out{0}{-6}	
	\foreach \x in {0, ..., 3}{
		\draw (\x-1.5, -4.5) circle (0.4);
		\draw[line] (\x-1.5, -5) -- (out\x);	
	}	

\end{tikzpicture}		
		\caption{{\bfseries CNN model 1 (kernel=6)*3} \\*
		}
		\label{fig:cnn_2_scheme}
	\end{subfigure}
	\begin{subfigure}[t]{0.3\linewidth}
		\begin{tikzpicture}
	\ohe{-0.1}{0.8}{6}
		\draw[ultra thick] (-1.3,0) rectangle (-0.7,0.8);
		\draw[ultra thick, ->] (-0.7, 0.4) -- (-0.1,0.4);
		
	\draw[line] (0-1, 0) -- (0-1,-1);
	\foreach \x in {1, 2}{
		\draw[line] (\x-1, -0.2) -- (\x-1,-1);
		
		}
	\foreach \x in {0, ..., 2}
		\node[cnn] (x) at (\x-1, -1.5) {c};
		
	\foreach \x in {0,..., 2}{
		\draw[line] (\x-1, -2.0) -- (\x-1,-2.3); }
		
	\oherandom{0}{-2.5}{4}{3}
	
	\flattenout{0}{-3.2}{12}
	
	\foreach \x in {0,..., 3}{
		\draw[line] (\x-1.5, -3.5) -- (\x-1.5,-4); }


	\out{0}{-6}	
	\foreach \x in {0, ..., 3}{
		\draw (\x-1.5, -4.5) circle (0.4);
		\draw[line] (\x-1.5, -5) -- (out\x);	
	}	

\end{tikzpicture}		
		\caption{{\bfseries CNN model 1 (kernel=3)*3 stride=3} \\* 
		}
		\label{fig:cnn_3_scheme}
	\end{subfigure}
	\caption{{\bfseries Архитектура некоторых сверточных моделей.} \\*   Простейшая сверточная модель состоит из трех сверточных нейронов, выход которых представляет из себя матрицу высотой 3, выпрямляющего слоя, четырех полносвязных нейронов с функцией активации softmax. \\
	(a) Минимальный размер фильтра $3\times4$ был выбран исходя из кодонной структуры б\'{о}льшей кодирующей части генома. \\  (b) Размер фильтра в сверточном слое может быть увеличен до $6\times4$, вследствие чего уменьшается размерность выхода и число параметров модели. \\ (c) Шаг сверточного фильтра может быть увеличен до 3, что может имитировать распознавание кодонной тринуклеотидной структуры кодирующих частей генома.
	}

	\label{fig:cnn_schemes}	
\end{figure*}
%
%\begin{figure*}[h] % picture
%	\centering
%	\begin{subfigure}[t]{0.4\textwidth}
%		\begin{tikzpicture}
	\oherandom{0}{1}{12}{4}
	\out{0}{-5.5}
	\draw[line](0, 0) -- (0, -1);
	\draw (-2,-2) rectangle (2, -1);
	\node (x) at (0, -1.5) {LSTM};
	\draw[line] (-2, -1.5) -- (-2.5, -1.5) -- (-2.5, -0.5) -- (2.5, -0.5) -- (2.5, -1.5) -- (2, -1.5) ;
	\draw[line](0, -2) -- (0, -2.5);
	
	\oherandom{0}{-2.5}{12}{1}
	\node (x) at (2.5, -2.5) {hidden states output};
	
	\foreach \x in {0, ..., 3}{
		\draw[line] (\x-1.5, -3) -- (\x-1.5,-3.5);
		\draw (\x-1.5, -4) circle (0.4);
		\draw[line] (\x-1.5, -4.5) -- (out\x);
		}
\end{tikzpicture}
%	\caption{} \label{fig:rnn}
%	\end{subfigure}
%	\begin{subfigure}[t]{0.5\textwidth}
%		\begin{tikzpicture}[
	scale = 0.9,
	every node/.style={scale=0.9},
    % GLOBAL CFG
    font=\sf \scriptsize,
    >=LaTeX,
    % Styles
    cell/.style={% For the main box
        rectangle, 
        rounded corners=5mm, 
        draw,
        very thick,
        },
    operator/.style={%For operators like +  and  x
        circle,
        draw,
        inner sep=-0.5pt,
        minimum height =.2cm,
        },
    function/.style={%For functions
        ellipse,
        draw,
        inner sep=1pt
        },
    ct/.style={% For external inputs and outputs
        circle,
        draw,
        line width = .75pt,
        minimum width=1cm,
        inner sep=1pt,
        },
    gt/.style={% For internal inputs
        rectangle,
        draw,
        minimum width=4mm,
        minimum height=3mm,
        inner sep=1pt
        },
    mylabel/.style={% something new that I have learned
        font=\scriptsize\rmfamily
        },
    mylabel1/.style={% something new that I have learned
        font=\normalsize	\rmfamily\bfseries
        },
    ArrowC1/.style={% Arrows with rounded corners
        rounded corners=.25cm,
        thick,
        },
    ArrowC2/.style={% Arrows with big rounded corners
        rounded corners=.5cm,
        thick,
        },
    ]

%Start drawing the thing...    
    % Draw the cell: 
    \node [cell, minimum height =4cm, minimum width=6cm] at (0,0){} ;

    % Draw inputs named ibox#
    \node [gt] (ibox1) at (-2,-0.75) {$\sigma$};
    \node [gt] (ibox2) at (-1.5,-0.75) {$\sigma$};
    \node [gt, minimum width=1cm] (ibox3) at (-0.5,-0.75) {Tanh};
    \node [gt] (ibox4) at (0.5,-0.75) {$\sigma$};
% Draw opérators   named mux# , add# and func#
    \node [operator] (mux1) at (-2,1.5) {$\times$};
    	\node[label={[mylabel1]left:f}] (f) at (-1.9, 0){};
    \node [operator] (add1) at (-0.5,1.5) {+};
    \node [operator] (mux2) at (-0.5,0) {$\times$};
    	\node[label={[mylabel1]left:i}] (i) at (-1.1, 0){};
    	\node[label={[mylabel1]left:$\mathbf{\tilde c}$}] (c) at (0.1,  -.4){};
    \node [operator] (mux3) at (1.5,0) {$\times$};
    	\node[label={[mylabel1]left:o}] (o) at (1, 0){};
    	\node[label={[mylabel1]left:c}] (c) at (1, 1.65){};
    \node [function] (func1) at (1.5,0.75) {Tanh};

    % Draw External inputs? named as basis c,h,x
    \node[ct, label={[mylabel]Prev. memory}] (c) at (-4,1.5) {\empt{c}{t-1}};
    \node[ct, label={[mylabel]Prev. hidden}] (h) at (-4,-1.5) {\empt{h}{t-1}};
    \node[ct, label={[mylabel]left:Input}] (x) at (-2.5,-3) {\empt{x}{t}};

    % Draw External outputs? named as basis c2,h2,x2
    \node[ct, label={[mylabel]Cur. memory}] (c2) at (4,1.5) {\empt{c}{t}};
    \node[ct, label={[mylabel]above:Cur. hidden}] (h2) at (4,-1.5) {\empt{h}{t}};
    \node[ct, label={[mylabel]left:Cur. hiddem as an output}] (x2) at (2.5,3) {\empt{h}{t}};
% Start connecting all.
    %Intersections and displacements are used. 
    % Drawing arrows    
    \draw [ArrowC1] (c) -- (mux1) -- (add1) -- (c2);

    % Inputs
    \draw [ArrowC2] (h) -| (ibox4);
    \draw [ArrowC1] (h -| ibox1)++(-0.5,0) -| (ibox1); 
    \draw [ArrowC1] (h -| ibox2)++(-0.5,0) -| (ibox2);
    \draw [ArrowC1] (h -| ibox3)++(-0.5,0) -| (ibox3);
    \draw [ArrowC1] (x) -- (x |- h)-| (ibox3);

    % Internal
    \draw [->, ArrowC2] (ibox1) -- (mux1);
    \draw [->, ArrowC2] (ibox2) |- (mux2);
    \draw [->, ArrowC2] (ibox3) -- (mux2);
    \draw [->, ArrowC2] (ibox4) |- (mux3);
    \draw [->, ArrowC2] (mux2) -- (add1);
    \draw [->, ArrowC1] (add1 -| func1)++(-0.5,0) -| (func1);
    \draw [->, ArrowC2] (func1) -- (mux3);
 %Outputs
    \draw [-, ArrowC2] (mux3) |- (h2);
    \draw (c2 -| x2) ++(0,-0.1) coordinate (i1);
    \draw [-, ArrowC2] (h2 -| x2)++(-0.5,0) -| (i1);
    \draw [-, ArrowC2] (i1)++(0,0.2) -- (x2);

\end{tikzpicture}
%	\caption{} \label{fig:lstm}
%	\end{subfigure}
%	\caption{{\bfseries Архитектура рекуррентной модели.} \\* }
%	\label{fig:rnn_scheme}	
%\end{figure*}


\begin{figure}
	\centering
	\begin{tikzpicture}
	\oherandom{0}{1}{12}{4}
	\out{0}{-5.5}
	\draw[line](0, 0) -- (0, -1);
	\draw (-2,-2) rectangle (2, -1);
	\node (x) at (0, -1.5) {LSTM};
	\draw[line] (-2, -1.5) -- (-2.5, -1.5) -- (-2.5, -0.5) -- (2.5, -0.5) -- (2.5, -1.5) -- (2, -1.5) ;
	\draw[line](0, -2) -- (0, -2.5);
	
	\oherandom{0}{-2.5}{12}{1}
	\node (x) at (2.5, -2.5) {hidden states output};
	
	\foreach \x in {0, ..., 3}{
		\draw[line] (\x-1.5, -3) -- (\x-1.5,-3.5);
		\draw (\x-1.5, -4) circle (0.4);
		\draw[line] (\x-1.5, -4.5) -- (out\x);
		}
\end{tikzpicture}
	\caption{{\bfseries Архитектура рекуррентной модели.} \\* }
	\label{fig:rnn_scheme}
\end{figure}
