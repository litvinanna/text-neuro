\begin{table}[h]
\label{table:baselines} \graytable
\caption{Точности предсказания, полученные различными математическими моделями.}
\newcolumntype{M}{>{$}l<{$}}
\begin{tabular}{l >{$}c<{$} >{$}l<{$}}
	Метод &  \text{Точность, \%} & \text{Число параметров} \\
	\thinrule
	Предсказание по частотам во всем геноме & 25.7 & 4\\
	 \verythinrule
	по частоте в предикторной области 12 нуклеотидов & 26.3 & 4\\
	\verythinrule
	по частоте в предикторной области 24 нуклеотидов & 26.5 & 4\\
	\verythinrule
	марковская модель 2 порядка & 30.3 &  4^2 = 16 \\
	\verythinrule
	марковская модель 3 порядка & 32.4 &  4^3 = 64\\
	\verythinrule
	марковская модель 4 порядка & 33.4 &  4^4 = 256\\
	\verythinrule
	марковская модель 5 порядка & 34.0 &  4^5 = 1,024\\
	\verythinrule
	марковская модель 6 порядка & 34.4 &  4^6 = 4,096\\
	\verythinrule
	марковская модель 7 порядка & 35.0 &  4^7 = 16,384\\
	\verythinrule
	марковская модель 8 порядка & 36.0 &  4^8 = 65,536\\
	\verythinrule
	марковская модель 9 порядка & 38.9 &  4^9 = 262,144\\
	\verythinrule
	марковская модель 10 порядка & 46.2 &  4^{10} = 1,048,576 \\
	\verythinrule
	марковская модель 11 порядка & 57.0 &  4^{11} \approx \text{геном }  E.coli  \\
	
\end{tabular}
	
\end{table}
